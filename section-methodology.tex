% =============================================================================
% METHODOLOGY SECTION
% =============================================================================
% This section describes the research methodology and approach.
% Replace the placeholder content with your actual methodology.

\section{Methodology}\label{sec:methodology}

Our approach builds upon the best practices identified in \cite{research2024} and incorporates the modular design principles outlined in \cite{latexguide2023}.

\subsection{Modular Document Structure}

Our approach involves organizing the LaTeX document into separate files for each major 
section, while maintaining a single main file that coordinates the overall structure. 
This modular approach provides several benefits:

\begin{enumerate}
    \item \textbf{Separation of Concerns}: Each section file contains only the content 
          relevant to that specific section.
    \item \textbf{Parallel Development}: Multiple authors can work on different sections 
          simultaneously without conflicts.
    \item \textbf{Easier Maintenance}: Individual sections can be modified without 
          affecting the entire document structure.
\end{enumerate}

\subsection{File Organization Strategy}

The document structure follows this organization:

\begin{verbatim}
main-acm2025.tex          # Main document with ACM metadata
|-- section-introduction.tex
|-- section-background.tex
|-- section-methodology.tex
|-- section-results.tex
`-- section-conclusion.tex
\end{verbatim}

\subsection{Implementation Details}

Each section file contains:
\begin{itemize}
    \item A clear section heading with appropriate labels
    \item Well-structured subsections
    \item Proper LaTeX formatting and citations
    \item No preamble or document-level commands
\end{itemize}

The main document handles:
\begin{itemize}
    \item Document class and package declarations
    \item Author and metadata information
    \item Abstract and acknowledgments
    \item Section inclusion via \texttt{\textbackslash input} commands
\end{itemize}

\subsection{Build Process}

The build process uses a Makefile that:
\begin{itemize}
    \item Compiles the main document with all included sections
    \item Runs LaTeX twice to resolve cross-references
    \item Outputs the final PDF to a build directory
    \item Provides clean and distclean targets for maintenance
\end{itemize} 