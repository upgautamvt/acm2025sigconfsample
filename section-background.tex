% =============================================================================
% BACKGROUND SECTION
% =============================================================================
% This section provides background information and related work.
% Replace the placeholder content with your actual background.

\section{Background}\label{sec:background}

\subsection{LaTeX Document Organization}

LaTeX has been the de facto standard for academic paper typesetting for several decades. 
Traditional LaTeX documents are typically organized as single files containing all sections, 
which works well for small documents but becomes problematic as document size and 
collaboration complexity increase.

\subsection{Related Work}

Several approaches have been proposed to address the challenges of large LaTeX document 
organization. Recent work by \cite{latex2023} has demonstrated the effectiveness of modular 
approaches using separate files for different document components, while other research 
has focused on collaborative editing workflows. However, these approaches often lack 
integration with specific journal or conference templates. The ACM template requirements 
\cite{acmwebsite2024} impose specific formatting constraints that must be maintained 
throughout the document.

\subsection{ACM Template Requirements}

The ACM template imposes specific formatting requirements that must be maintained 
throughout the document. These include:
\begin{itemize}
    \item Two-column layout for conference proceedings
    \item Specific font and spacing requirements
    \item Mandatory CCS concepts and keywords
    \item Structured author information and affiliations
\end{itemize}

\subsection{Version Control Challenges}

When multiple authors work on the same LaTeX document, version control systems often 
struggle with merge conflicts, particularly in complex environments like \texttt{amsmath} 
or \texttt{graphicx} packages. This leads to:
\begin{itemize}
    \item Increased compilation errors
    \item Time-consuming conflict resolution
    \item Potential loss of content during merges
\end{itemize} 